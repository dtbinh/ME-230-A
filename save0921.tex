The development of intelligent robots and autonomous vehicles is craving for real-time, safe, and stable motion planning in cluttered dynamic environments. For example, an automated guided vehicle (AGV) on factory floors \cite{wu2004modeling} needs to decide in real time how to bypass multiple human workers and a set of obstacles in the environment in order to approach its target efficiently \cite{wang2009autonomous,oleari2014industrial}. 

In literature, motion planning algorithms usually fall into three categories: graph-search based algorithm, sampling-based algorithm, and optimization-based algorithm. Among these algorithms, this paper focus on model predictive control (MPC) \cite{rawlings1999tutorial}, an optimization-based algorithm. MPC has been widely adopted both in academia research and in industrial applications. The fact that MPC controller observes the environment every time before solving the optimization problem allows the system to adjust and re-plan according to the changes of the environment. Its ability to handle input and state constraints also makes it popular in addressing motion planning problems. However, the existence of obstacles in the environment introduces non-convex state constraints, which results in non-convex MPC problems. It remains challenging to obtain real-time, safe and stable solutions for non-convex MPCs. An efficient optimization algorithm, i.e., the convex feasible set (CFS) algorithm \cite{liu2018convex}, has been proposed to obtain a safe open-loop trajectory in real time. However, even the trajectory calculated by CFS at each MPC time step is feasible and smooth, this cannot guarantee that the overall implemented trajectory is still smooth. Therefore, an analyzing method is needed to so that we can estimate whether the trajectory implemented by the MPC controller is smooth and desirable. The result of this analysis can also be serving as a stability property of the MPC controller.

Many efforts have been made in academia regarding stability analysis and closed-loop performance. Usually, stability can be guaranteed by modifying prediction horizon, adding terminal cost, adding terminal equality constraint, or using terminal set constraint instead \cite{mayne2000constrained}. These methods can guarantee that the calculated commands and the resulting states are bounded. However, these traditional techniques can only be applied to regulation problems. MPC motion planning problems are more complex than regulation problems \cite{limon2006stability}, which involves time-varying targets. On top of that,
even if the results are bounded, it does not guarantee that the results will be dynamically reasonable for the robot. Dynamically unreasonable trajectory will cause the robot to execute violent or zigzagging movement that is harmful to the robot's motors and also scares human in the environment.  

MPC with target changes has also been addressed in literature. One way of analyzing stability in such cases is to ensure feasibility, which is often sufficient to enable a simple guarantee of closed-loop stability for the controller \cite{boccia2014stability}\cite{dughman2015survey}\cite{zhang2016switched}. One common application of motion planning MPC problem is autonomous vehicles \cite{borrelli2005mpc}. In literature, stability is guaranteed by setting stability boundary for the control system, i.e., stability is quantified at several vehicle speeds. Generally, most of the stability analysis is still done by considering a regulation problem. Asymptotic stability can be guaranteed when feasibility is held and the cost-to-go is decreasing step by step.

However, as those methods only address convex MPC problems, it is difficult to apply them in the stability analysis of the non-convex MPC problem considered in this paper. Some work focused on MPC problems with non-convex cost function using the sequential convex optimization method \cite{hovgaard2013nonconvex}. Although the result is promising, the difficulty of ensuring closed-loop stability for non-convex MPC is also mentioned in the literature. 
Analyzing closed-loop stability of non-convex MPC problem remains challenging. A standard analytical approach to verify properties of non-convex MPC is still missing.


The contribution of this paper is to provide a new method to analyze the stability property of a non-convex MPC problem. In this paper, we first introduce a new notion of stability property called $M$-analysis and then analyze the properties of the proposed algorithm,  non-convex MPC that implements CFS (MPC-CFS). The relationships among the optimal trajectories generated at different steps will be examined, so as to guarantee that the robot trajectory will not experience sudden jumps during trajectory execution. The proposed method addresses the limitation of traditional stability analysis, and can further assure that the planning result is executable for the robot. Simulation studies are performed to test the performance of the implemented CFS  algorithm in the MPC framework. The stability features are analyzed using the proposed method. Finally, experiment is carried to verified the algorithm.

The remainder of the paper is organized as follows. Section II provides the problem formulation. Section III shows simulation results. Section IV shows the experiment result. Section V concludes the paper.